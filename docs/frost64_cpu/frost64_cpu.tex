\documentclass{article}

\usepackage{graphicx}
\usepackage{float}
\usepackage{fancyvrb}
\usepackage[T1]{fontenc}
\usepackage{lmodern}
\usepackage{setspace}
\usepackage[nottoc]{tocbibind}
\usepackage[font=large]{caption}
\usepackage{framed}
\usepackage{tabularx}
\usepackage{amsmath}
\usepackage{hyperref}
\usepackage{fontspec}
\usepackage[backend=biber,sorting=none]{biblatex}
%%\usepackage[
%%	backend=biber,
%%	style=ieee,
%%	sorting=none
%%]{biblatex}
%\addbibresource{project_refs.bib}

%% Hide section, subsection, and subsubsection numbering
%\renewcommand{\thesection}{}
%\renewcommand{\thesubsection}{}
%\renewcommand{\thesubsubsection}{}

%% Alternative form of doing section stuff
%\renewcommand{\thesection}{}
%\renewcommand{\thesubsection}{\arabic{section}.\arabic{subsection}}
%\makeatletter
%\def\@seccntformat#1{\csname #1ignore\expandafter\endcsname\csname the#1\endcsname\quad}
%\let\sectionignore\@gobbletwo
%\let\latex@numberline\numberline
%\def\numberline#1{\if\relax#1\relax\else\latex@numberline{#1}\fi}
%\makeatother
%
%\makeatletter
%\renewcommand\tableofcontents{%
%    \@starttoc{toc}%
%}
%\makeatother

\newcommand{\respacing}{\doublespacing \singlespacing}
\newcommand{\code}[2][1]{\noindent \texttt{\tab[#1] #2} \\}
\newcommand{\skipline}[0]{\texttt{}\\}
\newcommand{\tnl}[0]{\tabularnewline}

\begin{document}
%--------
	\font\titlefont={Times New Roman} at 20pt
	\title{{\titlefont Frost64 CPU}}

	\font\bottomtextfont={Times New Roman} at 12pt
	\date{{\bottomtextfont} \today}
	\author{{\bottomtextfont Andrew Clark}}

	\setmainfont{Times New Roman}
	\setmonofont{Courier New}

	\maketitle
	\pagenumbering{gobble}

	\newpage
	\pagenumbering{arabic}
	%\tableofcontents
	%\newpage

	%\doublespacing

%\section{Abstract}
	%\setcounter{section}{-1}

\section{Table of Contents}
	\tableofcontents
	\newpage

\section{Registers, Main Widths, etc.}
	\begin{itemize}
	%--------
	\item The main width of the processor, hinted at with the name, is
		64-bit.  Addresses are 64-bit, and some instructions only support
		64-bit operations.
	\item Instructions LARs (ILARs)
		\begin{itemize}
		%--------
		\item There are 256 total ILARs, with 128 of them used for
			interrupt service routines and 128 of them used for user code.
		\item The two ILARs named \texttt{i0}, which are ILAR number 0 and
			ILAR number 128 within the ILAR file, have their fields always
			set to zero.  They cannot be changed.
		\item ILARs are 256 bytes long, and instructions are 32-bit.  This
			means that one ILAR holds 256 / 4 = 64 instructions.
		%--------
		\end{itemize}
	\item Data LARs (DLARs)
		\begin{itemize}
		%--------
		\item There are 256 total DLARs, with 128 used for interrupt
			service routines and 128 of them used for user code.
		\item The two DLARs named \texttt{dzero}, which are DLAR number 0
			and DLAR number 128 within the DLAR file, have their fields
			always set to zero.  They cannot be changed.
		\item The user-mode DLAR named \texttt{dira}, which is DLAR 1
			within the DLAR file, is treated specially, as its address
			field is the interrupt return address for the \texttt{reti}
			instruction.
		\item DLARs are 256 bytes long.
		\item DLARs can take on the following type tags (3-bit enum):
			\begin{itemize}
			%--------
			\item 8-bit, unsigned
			\item 8-bit, signed
			\item 16-bit, unsigned
			\item 16-bit, signed
			\item 32-bit, unsigned
			\item 32-bit, signed
			\item 64-bit, unsigned
			\item 64-bit, signed
			%--------
			\end{itemize}
		\item The base address of a DLAR is 64 - 8 = 56 bits long.
		\item The scalar offset of a DLAR is 8 bits long.
		%--------
		\end{itemize}
	\item Program Counter (PC)
		\begin{itemize}
		%--------
		\item The program counter consists of an ILAR number (7-bit) and an
			offset into said ILAR (6-bit).
		\item Two program counters exist:  one for when servicing an
			interrupt, and one for when not servicing an interrupt (user
			mode).
		%--------
		\end{itemize}
	\item Interrupt Enable (\texttt{ie})
		\begin{itemize}
		%--------
		\item This 1-bit register indicates whether or not an interrupt can
			be serviced.
		%--------
		\end{itemize}
	%--------
	\end{itemize}

\section{Instructions (CPU's perspective)}
	\subsection{Group 0 Instructions}
		\begin{itemize}
		%--------
		\item Encoding:  \texttt{000v ooof fffa aaaa  aabb bbbb bccc cccc}
			\begin{itemize}
			%--------
			\item \texttt{v}:
				\begin{itemize}
				%--------
				\item when \texttt{0}:  scalar operation
				\item when \texttt{1}:  vector operation
				%--------
				\end{itemize}
			\item \texttt{o}:  opcode
			\item \texttt{f}:  functional unit number f
			\item \texttt{a}:  DLAR a
			\item \texttt{b}:  DLAR b
			\item \texttt{c}:  DLAR c
			%--------
			\end{itemize}
		\item Instruction List:
			\begin{enumerate}
			%--------
			\item \texttt{add fF, dA, dB, dC}
			\item \texttt{sub fF, dA, dB, dC}
			\item \texttt{slt fF, dA, dB, dC}
			\item \texttt{mul fF, dA, dB, dC}

			\item \texttt{div fF, dA, dB, dC}
				\begin{itemize}
				%--------
				\item Note:  Perform an unsigned divide if \texttt{dB} is
					unsigned, but perform a signed divide if \texttt{dB} is
					signed.
				%--------
				\end{itemize}
			\item \texttt{and fF, dA, dB, dC}
			\item \texttt{orr fF, dA, dB, dC}
			\item \texttt{xor fF, dA, dB, dC}
			%--------
			\end{enumerate}
		%--------
		\end{itemize}

	\subsection{Group 1 Instructions}
		\begin{itemize}
		%--------
		\item Encoding:  \texttt{001v ooof fffa aaaa  aabb bbbb bccc cccc}
			\begin{itemize}
			%--------
			\item \texttt{v}:
				\begin{itemize}
				%--------
				\item when \texttt{0}:  scalar operation
				\item when \texttt{1}:  vector operation
				%--------
				\end{itemize}
			\item \texttt{o}:  opcode
			\item \texttt{f}:  functional unit number f
			\item \texttt{a}:  DLAR a
			\item \texttt{b}:  DLAR b
			\item \texttt{c}:  DLAR c
			%--------
			\end{itemize}
		\item Instruction List:
			\begin{enumerate}
			%--------
			\item \texttt{shl fF, dA, dB, dC}
				\begin{itemize}
				%--------
				\item Shift left
				%--------
				\end{itemize}
			\item \texttt{shr fF, dA, dB, dC}
				\begin{itemize}
				%--------
				\item Shift right
				\item Note:  Perform a logical right shift if \texttt{dB}
					is unsigned, but perform an arithmetic right shift if
					\texttt{dB} is signed.
				%--------
				\end{itemize}
			\item Reserved
			\item Reserved

			\item Reserved
			\item Reserved
			\item Reserved
			\item Reserved
			%--------
			\end{enumerate}
		%--------
		\end{itemize}

	\subsection{Group 2 Instructions}
		\begin{itemize}
		%--------
		\item Encoding:  \texttt{010o oooa aaaa aabb  bbbb biii iiii iiii}
			\begin{itemize}
			%--------
			\item \texttt{o}:  opcode
			\item \texttt{a}:  DLAR a
			\item \texttt{b}:  DLAR b
			\item \texttt{i}:  Signed 11-bit immediate
			%--------
			\end{itemize}
		\item Notes:
			\begin{itemize}
			%--------
			\item These instructions compute the address to load from or
				store to via the following formula:
				\texttt{u64(dB.scalar\_data) + s64(simm11)}
			%--------
			\end{itemize}
		\item Instruction List:
			\begin{enumerate}
			%--------
			\item \texttt{ldiu8 dA, [dB, simm11]}
			\item \texttt{ldis8 dA, [dB, simm11]}
			\item \texttt{ldiu16 dA, [dB, simm11]}
			\item \texttt{ldis16 dA, [dB, simm11]}

			\item \texttt{ldiu32 dA, [dB, simm11]}
			\item \texttt{ldis32 dA, [dB, simm11]}
			\item \texttt{ldiu64 dA, [dB, simm11]}
			\item \texttt{ldis64 dA, [dB, simm11]}

			\item \texttt{stiu8 dA, [dB, simm11]}
			\item \texttt{stis8 dA, [dB, simm11]}
			\item \texttt{stiu16 dA, [dB, simm11]}
			\item \texttt{stis16 dA, [dB, simm11]}

			\item \texttt{stiu32 dA, [dB, simm11]}
			\item \texttt{stis32 dA, [dB, simm11]}
			\item \texttt{stiu64 dA, [dB, simm11]}
			\item \texttt{stis64 dA, [dB, simm11]}
			%--------
			\end{enumerate}
		%--------
		\end{itemize}

	\subsection{Group 3 Instructions}
		\begin{itemize}
		%--------
		\item Encoding:  \texttt{011o oooa aaaa aabb  bbbb bccc cccc iiii}
			\begin{itemize}
			%--------
			\item \texttt{o}:  opcode
			\item \texttt{a}:  DLAR a
			\item \texttt{b}:  DLAR b
			\item \texttt{b}:  DLAR c
			\item \texttt{i}:  Signed 4-bit immediate
			%--------
			\end{itemize}
		\item Notes:
			\begin{itemize}
			%--------
			\item These instructions compute the address to load from or
				store to via the following formula:
				\texttt{u64(dB.scalar\_data) + dC.addr + s64(simm4)}
			%--------
			\end{itemize}
		\item Instruction List:
			\begin{enumerate}
			%--------
			\item \texttt{ldu8 dA, [dB, dC, simm4]}
			\item \texttt{lds8 dA, [dB, dC, simm4]}
			\item \texttt{ldu16 dA, [dB, dC, simm4]}
			\item \texttt{lds16 dA, [dB, dC, simm4]}

			\item \texttt{ldu32 dA, [dB, dC, simm4]}
			\item \texttt{lds32 dA, [dB, dC, simm4]}
			\item \texttt{ldu64 dA, [dB, dC, simm4]}
			\item \texttt{lds64 dA, [dB, dC, simm4]}

			\item \texttt{stu8 dA, [dB, dC, simm4]}
			\item \texttt{sts8 dA, [dB, dC, simm4]}
			\item \texttt{stu16 dA, [dB, dC, simm4]}
			\item \texttt{sts16 dA, [dB, dC, simm4]}

			\item \texttt{stu32 dA, [dB, dC, simm4]}
			\item \texttt{sts32 dA, [dB, dC, simm4]}
			\item \texttt{stu64 dA, [dB, dC, simm4]}
			\item \texttt{sts64 dA, [dB, dC, simm4]}
			%--------
			\end{enumerate}
		%--------
		\end{itemize}

	\subsection{Group 4 Instructions}
		\begin{itemize}
		%--------
		\item Encoding:  \texttt{100a aaaa aabb bbbb  bccc cccc iiii jjjj}
			\begin{itemize}
			%--------
			\item \texttt{a}:  ILAR a (base ILAR)
			\item \texttt{b}:  DLAR b
			\item \texttt{c}:  ILAR c
			\item \texttt{i}:  Signed 4-bit immediate
			\item \texttt{j}:  Unsigned 4-bit immediate.  How many
				additional, consecutive ILARs, starting with \texttt{iA},
				will be \texttt{fetch}ed into.
				\begin{itemize}
				%--------
				\item Note:  The destination ILAR number will wrap around
					to \texttt{i0} if incrementing from \texttt{i128} to
					the next ILAR, i.e. a 7-bit counter is used.
				%--------
				\end{itemize}
			%--------
			\end{itemize}
		\item Notes:
			\begin{itemize}
			%--------
			\item The one instruction in this group of instructions,
				\texttt{fetch}, computes the starting address, via the
				following equation:  \texttt{u64(dB.data) + iC.addr
					+ s64(simm4)}
			%--------
			\end{itemize}
		\item Instruction List:
			\begin{itemize}
			%--------
			\item \texttt{fetch iA, dB, iC, simm4}
			%--------
			\end{itemize}
		%--------
		\end{itemize}

	\subsection{Group 5 Instructions}
		\begin{itemize}
		%--------
		\item Encoding:  \texttt{101o oooa aaaa aabb  bbbb biii iii0 0000}
			\begin{itemize}
			%--------
			\item \texttt{o}:  opcode
			\item \texttt{a}:  DLAR a (mode-specific)
			\item \texttt{b}:  LAR b (instruction-specific, either a DLAR
				or an ILAR)
			\item \texttt{i}:  6-bit immediate, \texttt{imm6}
			%--------
			\end{itemize}
		\item Instruction List:
			\begin{enumerate}
			%--------
			\item \texttt{gudata dA, dB, imm6}
				\begin{itemize}
				%--------
				\item Name:  "get user-mode data"
				\item For this instruction, the destination, mode-specific
					DLARs are treated as vectors of 128-bit data, where
					each vector element stores the following:
					\begin{itemize}
					%--------
					\item The full address (64-bit) of user-mode DLARs
						(starting with \texttt{dB}), stored in vector
						element bit range \texttt{63..0}.
					\item The type tag (3-bit) of user-mode DLARs (starting
						with \texttt{dB}), stored in vector element bit
						range \texttt{66..64}
					%--------
					\end{itemize}
				\item Each 128-bit vector element corresponds to one
					user-mode DLAR's contents.
				\item This instruction grabs the full address and type tag
					of user-mode DLARs, starting with \texttt{dB}, and
					including the following (consecutive) \texttt{imm6}
					user-mode DLARs, storing the 128-bit vector elements in
					(mode-specific) DLARs, starting with \texttt{dA} as the
					base DLAR.  The base mode-specific DLAR is followed by
					consecutive DLARs if needed to store the user-mode DLAR
					fields.
				\item If the last mode-specific DLAR is not completely
					filled with user-mode DLAR contents, its remaining
					128-bit vector elements will be filled with
					\texttt{0x0}.
				%--------
				\end{itemize}
			\item \texttt{rudata dA, dB, imm6}
				\begin{itemize}
				%--------
				\item Name:  "restore user-mode data"
				\item This instruction sort of does the reverse of the
					\texttt{gudata} instruction, performing the equivalent
					of load instructions, loading into user-mode DLARs
					based upon the metadata stored in mode-specific DLARs.
					As such, this instruction can read from memory, just
					like load instructions.
				%--------
				\end{itemize}

			\item \texttt{guinst dA, iB, imm6}
				\begin{itemize}
				%--------
				\item Name:  "get user-mode instructions"
				\item For this instruction, the destination, mode-specific
					DLARs are treated as vectors of 64-bit data, where each
					vector element stores the address field of a single
					user-mode ILAR.
				\item Each 64-bit vector element corresponds to one
					user-mode ILAR's address.
				\item This instruction grabs the full address of user-mode
					ILARs, starting with \texttt{dB} and including the
					following (consecutive) \texttt{imm6} user-mode ILARs,
					storing the 64-bit vector elements in mode-specific
					DLARs, starting with \texttt{dA} as the base DLAR.  The
					base mode-specific DLAR is followed by consecutive
					DLARs if needed to store the user-mode ILAR addresses.
				\item If the last mode-specific DLAR is not completely
					filled with user-mode ILAR addresses, its remaining
					64-bit vector elements will be filled with
					\texttt{0x0}.
				%--------
				\end{itemize}
			\item \texttt{ruinst dA, iB, imm6}
				\begin{itemize}
				%--------
				\item Name:  "restore user-mode instructions"
				\item This instruction performs \texttt{imm6 + 1}
					consecutive \texttt{fetch} operations, starting with
					(mode-specific) \texttt{dA} as the base register that
					is the source of addresses to \texttt{fetch}
					instructions from.  User-mode \texttt{iB} is the first
					ILAR that instructions will be \texttt{fetch}ed into.
				\item Basically, this instruction takes the results from
					a \texttt{guinst} instruction and re-\texttt{fetch}es
					the instructions back into user-mode ILARs.
				%--------
				\end{itemize}

			\item \texttt{gupc dA}
				\begin{itemize}
				%--------
				\item Name:  "get user-mode pc"
				\item This instruction grabs the user-mode program counter
					value, encoded into a scalar value as
					\begin{itemize}
					%--------
					\item \texttt{concat(ilar\_number, 
						ilar\_scalar\_offset)},
					%--------
					\end{itemize}
					and stores that value into DLAR \texttt{dA}.
				\item Note:  the type tag of mode-specific DLAR \texttt{dA}
					is respected during this operation.
				\item Note:  This instruction is useful for implementing
					user-mode jump tables that are (at least partially)
					stored in a bunch of ILARs.
				%--------
				\end{itemize}
			\item \texttt{rupc dA}
				\begin{itemize}
				%--------
				\item Name:  "restore user-mode pc"
				\item This instruction simply does the reverse operation of
					the \texttt{gupc} instruction.
				%--------
				\end{itemize}


			\item \texttt{gidata dA, dB, imm6}
				\begin{itemize}
				%--------
				\item Name:  "get interrupts-mode data"
				\item For this instruction, the destination, mode-specific
					DLARs are treated as vectors of 128-bit data, where
					each vector element stores the following:
					\begin{itemize}
					%--------
					\item The full address (64-bit) of interrupts-mode
						DLARs (starting with \texttt{dB}), stored in vector
						element bit range \texttt{63..0}.
					\item The type tag (3-bit) of interrupts-mode DLARs
						(starting with \texttt{dB}), stored in vector
						element bit range \texttt{66..64}
					%--------
					\end{itemize}
				\item Each 128-bit vector element corresponds to one
					interrupts-mode DLAR's contents.
				\item This instruction grabs the full address and type tag
					of interrupts-mode DLARs, starting with \texttt{dB},
					and including the following (consecutive) \texttt{imm6}
					interrupts-mode DLARs, storing the 128-bit vector
					elements in (mode-specific) DLARs, starting with
					\texttt{dA} as the base DLAR.  The base mode-specific
					DLAR is followed by consecutive DLARs if needed to
					store the interrupts-mode DLAR fields.
				\item If the last mode-specific DLAR is not completely
					filled with interrupts-mode DLAR contents, its
					remaining 128-bit vector elements will be filled with
					\texttt{0x0}.
				%--------
				\end{itemize}
			\item \texttt{ridata dA, dB, imm6}
				\begin{itemize}
				%--------
				\item Name:  "restore interrupts-mode data"
				\item This instruction sort of does the reverse of the
					\texttt{gidata} instruction, performing the equivalent
					of load instructions, loading into interrupts-mode DLARs
					based upon the metadata stored in mode-specific DLARs.
					As such, this instruction can read from memory, just
					like load instructions.
				%--------
				\end{itemize}

			\item \texttt{giinst dA, iB, imm6}
				\begin{itemize}
				%--------
				\item Name:  "get interrupts-mode instructions"
				\item For this instruction, the destination, mode-specific
					DLARs are treated as vectors of 64-bit data, where each
					vector element stores the address field of a single
					interrupts-mode ILAR.
				\item Each 64-bit vector element corresponds to one
					interrupts-mode ILAR's address.
				\item This instruction grabs the full address of
					interrupts-mode ILARs, starting with \texttt{dB} and
					including the following (consecutive) \texttt{imm6}
					interrupts-mode ILARs, storing the 64-bit vector
					elements in mode-specific DLARs, starting with
					\texttt{dA} as the base DLAR.  The base mode-specific
					DLAR is followed by consecutive DLARs if needed to
					store the interrupts-mode ILAR addresses.
				\item If the last mode-specific DLAR is not completely
					filled with interrupts-mode ILAR addresses, its
					remaining 64-bit vector elements will be filled with
					\texttt{0x0}.
				%--------
				\end{itemize}
			\item \texttt{riinst dA, iB, imm6}
				\begin{itemize}
				%--------
				\item Name:  "restore interrupts-mode instructions"
				\item This instruction performs \texttt{imm6 + 1}
					consecutive \texttt{fetch} operations, starting with
					(mode-specific) \texttt{dA} as the base register that
					is the source of addresses to \texttt{fetch}
					instructions from.  User-mode \texttt{iB} is the first
					ILAR that instructions will be \texttt{fetch}ed into.
				\item Basically, this instruction takes the results from
					a \texttt{giinst} instruction and re-\texttt{fetch}es
					the instructions back into interrupts-mode ILARs.
				%--------
				\end{itemize}

			\item \texttt{gipc dA}
				\begin{itemize}
				%--------
				\item Name:  "get interrupts-mode pc"
				\item This instruction grabs the interrupts-mode program
					counter value, encoded into a scalar value as
					\begin{itemize}
					%--------
					\item \texttt{concat(ilar\_number, 
						ilar\_scalar\_offset)},
					%--------
					\end{itemize}
					and stores that value into DLAR \texttt{dA}.
				\item Note:  the type tag of mode-specific DLAR \texttt{dA}
					is respected during this operation.
				\item Note:  This instruction is useful for implementing
					interrupts-mode jump tables that are (at least
					partially) stored in a bunch of ILARs.
				%--------
				\end{itemize}
			\item \texttt{ripc dA}
				\begin{itemize}
				%--------
				\item Name:  "restore interrupts-mode pc"
				\item This instruction simply does the reverse operation of
					the \texttt{gipc} instruction.
				%--------
				\end{itemize}


			\item \texttt{ei}
				\begin{itemize}
				%--------
				\item This instruction sets the interrupt enable register,
					\texttt{ie}, to \texttt{0b1}, which enables interrupts.
				%--------
				\end{itemize}
			\item \texttt{di}
				\begin{itemize}
				%--------
				\item This instruction sets the interrupt enable register,
					\texttt{ie}, to \texttt{0b0}, which disables
					interrupts.
				%--------
				\end{itemize}

			\item \texttt{reti dA}
				\begin{itemize}
				%--------
				\item Name:  "return from interrupt"
				\item This instruction does the following:
					\begin{itemize}
					%--------
					\item Sets \texttt{ie} to \texttt{0b1}, enabling
						interrupts.
					\item Switches to user-mode from interrupts-mode.
					\item Sets the user-mode \texttt{pc} to the scalar data
						of user-mode DLAR \texttt{dA}
					%--------
					\end{itemize}
				%--------
				\end{itemize}

			\item Reserved for future expansion
			%--------
			\end{enumerate}
		%--------
		\end{itemize}

	\subsection{Group 6 Instructions}
		\begin{itemize}
		%--------
		\item Encoding:  \texttt{110o ooaa aaaa abbb  bbbb iiii iijj jjjj}
			\begin{itemize}
			%--------
			\item \texttt{o}:  opcode
			\item \texttt{a}:  DLAR a, \texttt{dA}
			\item \texttt{b}:  ILAR b (\texttt{iB}), or DLAR b
				(\texttt{dB})
			\item \texttt{i}:  \texttt{i\_imm6}, field \texttt{i} 6-bit
				unsigned immediate
			\item \texttt{j}:  \texttt{j\_imm6}, field \texttt{j} 6-bit
				unsigned immediate
			%--------
			\end{itemize}
		\item Instruction List:
			\item \texttt{jzo.s dA, iB, i\_imm6}
				\begin{itemize}
				%--------
				\item Name:  "jump if zero, scalar"
				\item If \texttt{dA}'s \textit{scalar} data is zero, this
					instruction jumps to the instruction in ILAR
					\texttt{iB} at index \texttt{i\_imm6}.
				%--------
				\end{itemize}
			\item \texttt{jzo.v dA, iB, i\_imm6}
				\begin{itemize}
				%--------
				\item Name:  "jump if zero, vector"
				\item If \texttt{dA}'s \textit{entire} (vector) data is
					zero, this instruction jumps to the instruction in ILAR
					\texttt{iB} at index \texttt{i\_imm6}.
				%--------
				\end{itemize}

			\item \texttt{jnz.s dA, iB, i\_imm6}
				\begin{itemize}
				%--------
				\item Name:  "jump if non-zero, scalar"
				\item If \texttt{dA}'s \textit{scalar} data is non-zero,
					this instruction jumps to the instruction in ILAR
					\texttt{iB} at index \texttt{i\_imm6}.
				%--------
				\end{itemize}
			\item \texttt{jnz.v dA, iB, i\_imm6}
				\begin{itemize}
				%--------
				\item Name:  "jump if non-zero, vector"
				\item If \texttt{dA}'s \textit{entire} (vector) data is
					non-zero, this instruction jumps to the instruction in
					ILAR \texttt{iB} at index \texttt{i\_imm6}.
				%--------
				\end{itemize}

			\item \texttt{sel.s dA, iB, i\_imm6, j\_imm6}
				\begin{itemize}
				%--------
				\item Name:  "select, scalar"
				\item If \texttt{dA}'s scalar data is non-zero, jump to the
					instruction at ILAR \texttt{iB}, offset
					\texttt{i\_imm6}.  Otherwise, if \texttt{dA}'s scalar
					data is zero, jump to the instruction at ILAR
					\texttt{iB}, offset \texttt{j\_imm6}.
				%--------
				\end{itemize}

			\item \texttt{sel.v dA, iB, i\_imm6, j\_imm6}
				\begin{itemize}
				%--------
				\item Name:  "select, vector"
				\item If \texttt{dA}'s \textit{entire} (vector) data is
					non-zero, jump to the instruction at ILAR \texttt{iB},
					offset \texttt{i\_imm6}.  Otherwise, if \texttt{dA}'s
					\textit{entire} (vector) data is zero, jump to the
					instruction at ILAR \texttt{iB}, offset
					\texttt{j\_imm6}.
				%--------
				\end{itemize}

			\item Reserved for future expansion.
			\item Reserved for future expansion.
		%--------
		\end{itemize}

	%\printbibliography[heading=bibnumbered,title={Bibliography}]

%--------
\end{document}
