\documentclass{article}

\usepackage{graphicx}
\usepackage{float}
\usepackage{fancyvrb}
\usepackage[T1]{fontenc}
\usepackage{lmodern}
\usepackage{setspace}
\usepackage[nottoc]{tocbibind}
\usepackage[font=large]{caption}
\usepackage{framed}
\usepackage{tabularx}
\usepackage{amsmath}
\usepackage{hyperref}
\usepackage{fontspec}
\usepackage[backend=biber,sorting=none]{biblatex}
%%\usepackage[
%%	backend=biber,
%%	style=ieee,
%%	sorting=none
%%]{biblatex}
%\addbibresource{project_refs.bib}

%% Hide section, subsection, and subsubsection numbering
%\renewcommand{\thesection}{}
%\renewcommand{\thesubsection}{}
%\renewcommand{\thesubsubsection}{}

%% Alternative form of doing section stuff
%\renewcommand{\thesection}{}
%\renewcommand{\thesubsection}{\arabic{section}.\arabic{subsection}}
%\makeatletter
%\def\@seccntformat#1{\csname #1ignore\expandafter\endcsname\csname the#1\endcsname\quad}
%\let\sectionignore\@gobbletwo
%\let\latex@numberline\numberline
%\def\numberline#1{\if\relax#1\relax\else\latex@numberline{#1}\fi}
%\makeatother
%
%\makeatletter
%\renewcommand\tableofcontents{%
%    \@starttoc{toc}%
%}
%\makeatother

\newcommand{\respacing}{\doublespacing \singlespacing}
\newcommand{\code}[2][1]{\noindent \texttt{\tab[#1] #2} \\}
\newcommand{\skipline}[0]{\texttt{}\\}
\newcommand{\tnl}[0]{\tabularnewline}

\begin{document}
%--------
	\font\titlefont={Times New Roman} at 20pt
	\title{{\titlefont Frost64 GPU}}

	\font\bottomtextfont={Times New Roman} at 12pt
	\date{{\bottomtextfont} \today}
	\author{{\bottomtextfont Andrew Clark}}

	\setmainfont{Times New Roman}
	\setmonofont{Courier New}

	\maketitle
	\pagenumbering{gobble}

	\newpage
	\pagenumbering{arabic}
	%\tableofcontents
	%\newpage

	%\doublespacing

%\section{Abstract}
	%\setcounter{section}{-1}

\section{Table of Contents}
	\tableofcontents
	\newpage

\section{Registers, Main Widths, etc.}
	\begin{itemize}
	%--------
	\item The main width of the processor, hinted at with the name, is
		64-bit.  Addresses are 64-bit, and some instructions only support
		64-bit operations.
	\item Instructions LARs (ILARs)
		\begin{itemize}
		%--------
		\item There are 128 total ILARs, with 64 of them used for
			supervisor mode and 64 of them used for user mode.
		\item The two ILARs named \texttt{i0}, which are ILAR number 0 and
			ILAR number 64 within the ILAR file, have their fields always
			set to zero.  They cannot be changed.
		\item ILARs are 256 bytes long, and instructions are 32-bit.  This
			means that one ILAR holds 256 / 4 = 64 instructions.
		\item Note that supervisor mode ILARs are neither readable nor
			writable in user mode.
		%--------
		\end{itemize}
	\item Data LARs (DLARs)
		\begin{itemize}
		%--------
		\item There are 128 total DLARs, with 64 used for supervisor mode
			and 64 of them used for user mode.
		\item The two DLARs named \texttt{dzero}, which are DLAR number 0
			and DLAR number 128 within the DLAR file, have their fields
			always set to zero.  They cannot be changed.
		\item The supervisor mode DLAR named \texttt{dira}, which is DLAR
			126 within the DLAR file (index \texttt{64} when in supervisor
			mode), is treated specially, as its scalar data field is the
			interrupt return address for the \texttt{reti} instruction.
		\item The supervisor mode DLAR named \texttt{dida}, which is DLAR
			127 within the DLAR file (index \texttt{65} when in supervisor
			mode), is used as the interrupt vector.
		\item DLARs are 256 bytes long.
		\item DLARs can take on the following type tags (3-bit enum):
			\begin{itemize}
			%--------
			\item 8-bit, unsigned
			\item 8-bit, signed
			\item 16-bit, unsigned
			\item 16-bit, signed
			\item 32-bit, unsigned
			\item 32-bit, signed
			\item 64-bit, unsigned
			\item 64-bit, signed
			%--------
			\end{itemize}
		\item The base address of a DLAR is 64 - 8 = 56 bits long.
		\item The scalar offset of a DLAR is 8 bits long.
		%--------
		\end{itemize}
	\item Program Counter (PC)
		\begin{itemize}
		%--------
		\item The program counter consists of an ILAR number (6-bit) and an
			offset into said ILAR (6-bit).
		\item Two program counters exist:  one for supervisor mode, and one
			for user mode.
		%--------
		\end{itemize}
	\item Interrupt Enable (\texttt{ie})
		\begin{itemize}
		%--------
		\item This 1-bit register indicates whether or not an interrupt can
			be serviced.  It can only be written to in supervisor mode, and
			it cannot be read at all.
		%--------
		\end{itemize}
	%--------
	\end{itemize}

%\newpage
%\section{Encoding Functional Unit Numbers Into Arithmetic/Logic
%	Instructions}
%
%	Here is an explanation of what encoding functional unit numbers into
%	arithmetic/logic instructions means to the CPU.
%
%	"This instruction, which takes multiple cycles to complete, uses this
%	functional unit.  You may execute later instructions that don't depend
%	on the results that this functional unit will produce at the same time
%	as this instruction.  If a later instruction needs the results of this
%	functional unit, please stall, as necessary, until the functional unit
%	has produced its results."
%
%	In other words, said instructions start an asynchronous use of the
%	functional unit, stalling only if necessary, i.e. when a needed result
%	isn't computed yet.
%
%	Different implementations of the CPU may have a different number of
%	functional units, with the machine still executing instructions
%	correctly, even if it has fewer functional units than what was encoded
%	into the instruction.  Four bits in relevant instructions are used to
%	indicate which functional unit to use for said instructions.

\newpage
\section{Exceptions}
	Some instructions may cause an exception to occur, putting the
	processor in supervisor mode.

	The following exceptions may occur during normal execution of a
	program.

	\begin{itemize}
	%--------
	\item Division by zero.
	\item \texttt{swi}.
		\begin{itemize}
		%--------
		\item Note that this instruction always causes an exception to
			occur.
		%--------
		\end{itemize}
	\item \texttt{reti} in user mode.
	\item \texttt{ei} in user mode.
	\item \texttt{di} in user mode.
	%--------
	\end{itemize}

\newpage
\section{Instructions (CPU's perspective)}
	\subsection{Group 0 Instructions}
		\begin{itemize}
		%--------
		\item Encoding:  \texttt{0000 aaaa aabb bbbb  cccc cc00 000v oooo}
			\begin{itemize}
			%--------
			\item \texttt{a}:  DLAR a
			\item \texttt{b}:  DLAR b
			\item \texttt{c}:  DLAR c
			\item \texttt{v}:
				\begin{itemize}
				%--------
				\item when \texttt{0}:  scalar operation
				\item when \texttt{1}:  vector operation
				%--------
				\end{itemize}
			\item \texttt{o}:  opcode
			%--------
			\end{itemize}
		\item Instruction List:
			\begin{enumerate}
			%--------
			\item \texttt{add dA, dB, dC}
			\item \texttt{sub dA, dB, dC}
			\item \texttt{slt dA, dB, dC}
				\begin{itemize}
				%--------
				\item Perform an unsigned set less than if \texttt{dB} is
					unsigned, but perform a signed set less than if
					\texttt{dB} is signed.
				%--------
				\end{itemize}
			\item \texttt{mul dA, dB, dC}


			\item \texttt{div dA, dB, dC}
				\begin{itemize}
				%--------
				\item Perform an unsigned divide if \texttt{dB} is
					unsigned, but perform a signed divide if \texttt{dB} is
					signed.  This instruction causes an exception if
					division by zero is attempted.
				%--------
				\end{itemize}
			\item \texttt{and dA, dB, dC}
			\item \texttt{orr dA, dB, dC}
			\item \texttt{xor dA, dB, dC}


			\item \texttt{shl dA, dB, dC}
				\begin{itemize}
				%--------
				\item Logical shift left
				%--------
				\end{itemize}
			\item \texttt{shr dA, dB, dC}
				\begin{itemize}
				%--------
				\item Logical shift right if \texttt{dB} is unsigned, but
					arithmetic shift right if \texttt{dB} is signed.
				%--------
				\end{itemize}
			\item \texttt{rol dA, dB, dC}
			\item \texttt{ror dA, dB, dC}


			\item \textit{Reserved for future expansion.}
			\item \textit{Reserved for future expansion.}
			\item \textit{Reserved for future expansion.}
			\item \textit{Reserved for future expansion.}
			%--------
			\end{enumerate}
		%--------
		\end{itemize}

	\subsection{Group 1 Instructions}
		\begin{itemize}
		%--------
		\item Encoding:  \texttt{0001 aaaa aabb bbbb  iiii iiii iiii oooo}
			\begin{itemize}
			%--------
			\item \texttt{a}:  DLAR a
			\item \texttt{b}:  DLAR b
			%\item \texttt{c}:  DLAR c
			\item \texttt{i}:  signed 12-bit immediate, \texttt{simm12}
			\item \texttt{o}:  opcode
			%--------
			\end{itemize}
		\item Notes:
			These instructions compute the address to load from or store to
			via the formula \texttt{dB.scalar\_data + to\_s64(simm12)}
		\item Instruction List:
			\begin{enumerate}
			%--------
			\item \texttt{ldu8 dA, [dB, simm12]}
			\item \texttt{lds8 dA, [dB, simm12]}
			\item \texttt{ldu16 dA, [dB, simm12]}
			\item \texttt{lds16 dA, [dB, simm12]}

			\item \texttt{ldu32 dA, [dB, simm12]}
			\item \texttt{lds32 dA, [dB, simm12]}
			\item \texttt{ldu64 dA, [dB, simm12]}
			\item \texttt{lds64 dA, [dB, simm12]}

			\item \texttt{stu8 dA, [dB, simm12]}
			\item \texttt{sts8 dA, [dB, simm12]}
			\item \texttt{stu16 dA, [dB, simm12]}
			\item \texttt{sts16 dA, [dB, simm12]}

			\item \texttt{stu32 dA, [dB, simm12]}
			\item \texttt{sts32 dA, [dB, simm12]}
			\item \texttt{stu64 dA, [dB, simm12]}
			\item \texttt{sts64 dA, [dB, simm12]}
			%--------
			\end{enumerate}
		%--------
		\end{itemize}

	\subsection{Group 2 Instructions}
		\begin{itemize}
		%--------
		\item Encoding:  \texttt{0010 aaaa aabb bbbb  cccc ccii iiii oooo}
			\begin{itemize}
			%--------
			\item \texttt{a}:  DLAR a
			\item \texttt{b}:  DLAR b
			\item \texttt{c}:  DLAR c
			\item \texttt{i}:  signed 6-bit immediate, \texttt{simm6}
			\item \texttt{o}:  opcode
			%--------
			\end{itemize}
		\item Notes:
			These instructions compute the address to load from or store to
			via the formula
			\texttt{dB.scalar\_data + dC.addr + to\_s64(simm12)}
		\item Instruction List:
			\begin{enumerate}
			%--------
			\item \texttt{ldau8 dA, [dB, dC, simm6]}
			\item \texttt{ldas8 dA, [dB, dC, simm6]}
			\item \texttt{ldau16 dA, [dB, dC, simm6]}
			\item \texttt{ldas16 dA, [dB, dC, simm6]}

			\item \texttt{ldau32 dA, [dB, dC, simm6]}
			\item \texttt{ldas32 dA, [dB, dC, simm6]}
			\item \texttt{ldau64 dA, [dB, dC, simm6]}
			\item \texttt{ldas64 dA, [dB, dC, simm6]}

			\item \texttt{stau8 dA, [dB, dC, simm6]}
			\item \texttt{stas8 dA, [dB, dC, simm6]}
			\item \texttt{stau16 dA, [dB, dC, simm6]}
			\item \texttt{stas16 dA, [dB, dC, simm6]}

			\item \texttt{stau32 dA, [dB, dC, simm6]}
			\item \texttt{stas32 dA, [dB, dC, simm6]}
			\item \texttt{stau64 dA, [dB, dC, simm6]}
			\item \texttt{stas64 dA, [dB, dC, simm6]}
			%--------
			\end{enumerate}
		%--------
		\end{itemize}

	\subsection{Group 3 Instructions}
		\begin{itemize}
		%--------
		\item Encoding:  \texttt{0011 aaaa aabb bbbb  cccc ccii iiii jjjj}
			\begin{itemize}
			%--------
			\item \texttt{a}:  ILAR a
			\item \texttt{b}:  DLAR b
			\item \texttt{c}:  ILAR c
			\item \texttt{i}:  signed 6-bit immediate, \texttt{isimm6}
			\item \texttt{j}:  unsigned 4-bit immediate, \texttt{jimm4}
			%--------
			\end{itemize}
		\item Notes:
			\begin{itemize}
			%--------
			\item This instruction computes the base address to fetch from 
				via the formula
				\texttt{dB.scalar\_data + iC.addr + to\_s64(isimm6)}
			\item This instruction fetches into up to \texttt{jimm4}
				consecutive ILARs, starting with \texttt{iA}, allowing a
				maximum 17 ILARs to be fetched into with one instruction.
			\item This instruction makes use of the associativity of LARs
				and will only actually access memory if it has to.
			%--------
			\end{itemize}
			
		\item Instruction List:
			\begin{enumerate}
			%--------
			\item \texttt{fetch iA, [dB, iC, isimm6], jimm4}
			%--------
			\end{enumerate}
		%--------
		\end{itemize}

	\subsection{Group 4 Instructions}
		\begin{itemize}
		%--------
		\item Encoding:  \texttt{0100 aaaa aabb bbbb  cccc ccii iiii oooo}
			\begin{itemize}
			%--------
			\item \texttt{a}:  DLAR a
			\item \texttt{b}:  DLAR b or ILAR b
			\item \texttt{c}:  ILAR c or unsigned 6-bit immediate,
				\texttt{cimm6}
			\item \texttt{i}:  unsigned 6-bit immediate, \texttt{iimm6},
				or signed 6-bit immediate, \texttt{isimm6}
			%--------
			\end{itemize}
		\item Instruction List:
			\begin{enumerate}
			%--------

			%--------
			\item \texttt{sel.s dA, iB, iC, isimm6}
				\begin{itemize}
				%--------
				\item This instruction jumps to \texttt{iB}, offset
					\texttt{imm6}, if \texttt{dA.scalar\_data} is zero or
					\texttt{iC}, offset \texttt{imm6} if
					\texttt{dA.scalar\_data} is non-zero.
				%--------
				\end{itemize}
			\item \texttt{sel.v dA, iB, iC, isimm6}
				\begin{itemize}
				%--------
				\item This instruction jumps to \texttt{iB}, offset
					\texttt{imm6}, if \texttt{dA.data} is zero or
					\texttt{iC}, offset \texttt{imm6} if
					\texttt{dA.data} is non-zero.
				%--------
				\end{itemize}
			\item \texttt{bz.s dA, iB, isimm6}
			\item \texttt{bz.v dA, iB, isimm6}
			%--------

			%--------
			\item \texttt{bnz.s dA, iB, isimm6}
			\item \texttt{bnz.v dA, iB, isimm6}
			\item \texttt{in.s dA, dB, cimm6, iimm6}
				\begin{itemize}
				%--------
				\item Input scalar data into \texttt{dA} from IO address in
					\texttt{dB.scalar\_data + cat(cimm6, iimm6)}.
				\item The size of the IO access is determined by
					\texttt{dA.type}.
				%--------
				\end{itemize}
			\item \texttt{in.v, dA, dB, cimm6, iimm6}
				\begin{itemize}
				%--------
				\item Input vector data into \texttt{dA} from IO address in
					\texttt{dB.scalar\_data + cat(cimm6, iimm6)}.
				%--------
				\end{itemize}
			%--------

			%--------
			\item \texttt{out.s dA, dB, cimm6, iimm6}
				\begin{itemize}
				%--------
				\item Output scalar data in \texttt{dA} to IO address in
					\texttt{dB.scalar\_data + cat(cimm6, iimm6)}.
				\item The size of the IO access is determined by
					\texttt{dA.type}.
				%--------
				\end{itemize}
			\item \texttt{out.v, dA, dB, cimm6, iimm6}
				\begin{itemize}
				%--------
				\item Output vector data in \texttt{dA} to IO address in
					\texttt{dB.scalar\_data + cat(cimm6, iimm6)}.
				%--------
				\end{itemize}
			\item \texttt{swi dA, cimm6, iimm6}
				\begin{itemize}
				%--------
				\item Perform software interrupt, where the interrupt
					number is stored in
					\texttt{dA.scalar\_data + cat(cimm6, iimm6)}
				\item This instruction, like a regular interrupt, stores
					the interrupt return address in \texttt{dira}, which is
					a supervisor mode DLAR.
				%--------
				\end{itemize}
			\item \texttt{reti}
				\begin{itemize}
				%--------
				\item Switch from supervisor mode to user mode.  This
					instruction will cause an exception if executed in user
					mode. 
				%--------
				\end{itemize}
			%--------

			%--------
			\item \texttt{ei}
				\begin{itemize}
				%--------
				\item Enable interrupts.  This instruction causes an
					exception if executed in user mode.
				%--------
				\end{itemize}
			\item \texttt{di}
				\begin{itemize}
				%--------
				\item Disable interrupts.  This instruction causes an
					exception if executed in user mode.
				%--------
				\end{itemize}
			\item \textit{Reserved for future expansion}
			\item \texttt{Reserved for future expansion.}
			%--------

			%--------
			\end{enumerate}
		%--------
		\end{itemize}


	%\printbibliography[heading=bibnumbered,title={Bibliography}]

%--------
\end{document}
