\documentclass{article}

\usepackage{pbox}
\usepackage{graphicx}
\usepackage{float}
\usepackage{fancyvrb}
\usepackage[T1]{fontenc}
\usepackage{lmodern}
\usepackage{setspace}
\usepackage[nottoc]{tocbibind}
\usepackage[font=large]{caption}
\usepackage{framed}
\usepackage{tabularx}
\usepackage{amsmath}
\usepackage{hyperref}
\usepackage{fontspec}
\usepackage[backend=biber,sorting=none]{biblatex}
%%\usepackage[
%%	backend=biber,
%%	style=ieee,
%%	sorting=none
%%]{biblatex}
%\addbibresource{project_refs.bib}

%% Hide section, subsection, and subsubsection numbering
%\renewcommand{\thesection}{}
%\renewcommand{\thesubsection}{}
%\renewcommand{\thesubsubsection}{}

%% Alternative form of doing section stuff
%\renewcommand{\thesection}{}
%\renewcommand{\thesubsection}{\arabic{section}.\arabic{subsection}}
%\makeatletter
%\def\@seccntformat#1{\csname #1ignore\expandafter\endcsname\csname the#1\endcsname\quad}
%\let\sectionignore\@gobbletwo
%\let\latex@numberline\numberline
%\def\numberline#1{\if\relax#1\relax\else\latex@numberline{#1}\fi}
%\makeatother
%
%\makeatletter
%\renewcommand\tableofcontents{%
%    \@starttoc{toc}%
%}
%\makeatother

\newcommand{\respacing}{\doublespacing \singlespacing}
\newcommand{\code}[2][1]{\noindent \texttt{\tab[#1] #2} \\}
\newcommand{\skipline}[0]{\texttt{}\\}
\newcommand{\tnl}[0]{\tabularnewline}

\begin{document}
%--------
	\font\titlefont={Times New Roman} at 20pt
	\title{{\titlefont Frost64 Utility CPU}}

	\font\bottomtextfont={Times New Roman} at 12pt
	\date{{\bottomtextfont} \today}
	\author{{\bottomtextfont Andrew Clark}}

	\setmainfont{Times New Roman}
	\setmonofont{Courier New}

	\maketitle
	\pagenumbering{gobble}

	\newpage
	\pagenumbering{arabic}
	%\tableofcontents
	%\newpage

	%\doublespacing

%\section{Abstract}
	%\setcounter{section}{-1}

\section{Table of Contents}
	\tableofcontents
	\newpage

\section{Registers, Main Widths, etc.}
	\begin{itemize}
	%--------
	\item The main width of the processor is 32-bit, contrary to what the
		name of the processor suggests.  Addresses are 32-bit. 
	\item General Purpose Registers (32-bit)
		There are sixteen general-purpose registers:
		\begin{itemize}
		%--------
		\item \texttt{zero}: index \texttt{0x0}, always zero
		\item \texttt{u0}: index \texttt{0x1}
		\item \texttt{u1}: index \texttt{0x2}
		\item \texttt{u2}: index \texttt{0x3}
		\item \texttt{u3}: index \texttt{0x4}
		\item \texttt{u4}: index \texttt{0x5}
		\item \texttt{u5}: index \texttt{0x6}
		\item \texttt{u6}: index \texttt{0x7}
		\item \texttt{u7}: index \texttt{0x8}
		\item \texttt{u8}: index \texttt{0x9}
		\item \texttt{u9}: index \texttt{0xa}
		\item \texttt{u10}: index \texttt{0xb}
		\item \texttt{u11}: index \texttt{0xc}
		\item \texttt{lr}: index \texttt{0xd}, link register
		\item \texttt{fp}: index \texttt{0xe}, frame pointer
		\item \texttt{sp}: index \texttt{0xf}, stack pointer
		%--------
		\end{itemize}

		There are a few special-purpose registers:
		\begin{itemize}
		%--------
		\item \texttt{pc}: index \texttt{0x0}, program counter
			\begin{itemize}
			%--------
			\item Note that writes into this register forcibly have the LSB
				set to \texttt{0b0}.
			%--------
			\end{itemize}
		\item \texttt{ira}: index \texttt{0x1}, interrupt return address
		\item \texttt{ida}: index \texttt{0x2}, interrupt destination
			address
		\item \texttt{inp}: index \texttt{0x3}, interrupt pin (which
			interrupt pin was asserted)
		\item \texttt{ie}: index \texttt{0x4}, interrupt enable
		%--------
		\end{itemize}
	%--------
	\end{itemize}

\section{Instructions}
	\subsection{Group 0 Instructions}
		\begin{itemize}
		%--------
		\item Encoding:  \texttt{0ooo aaaa iiii iiii}
			\begin{itemize}
			%--------
			\item \texttt{o}:  opcode
			\item \texttt{a}:  general-purpose register \texttt{rA}
			\item \texttt{i}:  8-bit unsigned immediate \texttt{imm8} or
				8-bit signed immediate \texttt{simm8}
			%--------
			\end{itemize}
		\item Instruction List:
			\begin{enumerate}
			%--------
			\item \texttt{cpyi rA, imm8}
				\begin{itemize}
				%--------
				\item This instruction sets \texttt{rA} to the immediate
					value.
				%--------
				\end{itemize}
			\item \texttt{shiftin rA, imm8}
				\begin{itemize}
				%--------
				\item This instruction sets \texttt{rA} to
					\texttt{(rA << 8) | imm8}
				%--------
				\end{itemize}
			\item \texttt{addsi rA, simm8}
				\begin{itemize}
				%--------
				\item This instruction writes \texttt{rA + to\_s32(simm8)}
					into \texttt{rA}.
				%--------
				\end{itemize}
			\item \texttt{sltui rA, simm8}

			\item \texttt{sltsi rA, simm8}
			\item \texttt{addsi rA, pc, simm8}
				\begin{itemize}
				%--------
				\item This instruction writes \texttt{pc + to\_s32(imm8)}
					into \texttt{rA}
				%--------
				\end{itemize}

			\item \texttt{bz rA, simm8}
				\begin{itemize}
				%--------
				\item Relative branch if \texttt{rA} is \textit{zero}.
				\item Note that the branch offset is treated as
					\texttt{simm8 << 1}.
				%--------
				\end{itemize}
			\item \texttt{bnz rA, simm8}
				\begin{itemize}
				%--------
				\item Relative branch if \texttt{rA} is \textit{non-zero}.
				\item Note that the branch offset is treated as
					\texttt{simm8 << 1}.
				%--------
				\end{itemize}
			%--------
			\end{enumerate}
		%--------
		\end{itemize}

	\subsection{Group 1 Instructions}
		\begin{itemize}
		%--------
		\item Encoding:  \texttt{1000 aaaa oooi iiii}
			\begin{itemize}
			%--------
			\item \texttt{a}:  general-purpose register \texttt{rA} or
				special-purpose register \texttt{sA}
			\item \texttt{o}:  opcode
			\item \texttt{i}:  5-bit unsigned immediate \texttt{imm5} or
				5-bit signed immediate \texttt{simm5}
			%--------
			\end{itemize}
		\item Instruction List:
			\begin{enumerate}
			%--------
			\item \texttt{xorsi rA, simm5}
			\item \texttt{xorsi sA, simm5}
			\item \texttt{lsli rA, imm5}
			\item \texttt{lsri rA, imm5}

			\item \texttt{asri rA, imm5}
			\item \textit{Reserved for future expansion.}
			\item \textit{Reserved for future expansion.}
			\item \textit{Reserved for future expansion.}
			%--------
			\end{enumerate}
		%--------
		\end{itemize}

	\subsection{Group 2 Instructions}
		\begin{itemize}
		%--------
		\item Encoding:  \texttt{1001 aaaa bbbb oooo}
			\begin{itemize}
			%--------
			\item \texttt{a}:  general-purpose register \texttt{rA}
			\item \texttt{b}:  general-purpose register \texttt{rB}
			\item \texttt{o}:  opcode
			%--------
			\end{itemize}
		\item Instruction List:
			\begin{enumerate}
			%--------
			\item \texttt{add rA, rB}
			\item \texttt{sub rA, rB}
			\item \texttt{sltu rA, rB}
			\item \texttt{slts rA, rB}

			\item \texttt{and rA, rB}
			\item \texttt{orr rA, rB}
			\item \texttt{xor rA, rB}
			\item \texttt{lsl rA, rB}

			\item \texttt{lsr rA, rB}
			\item \texttt{asr rA, rB}
			\item \texttt{mul rA, rB}
			\item \texttt{udiv rA, rB}

			\item \texttt{sdiv rA, rB}
			\item \textit{add rA, rB, fp}
			\item \textit{add rA, rB, sp}
			\item \textit{add rA, rB, pc}
			%--------
			\end{enumerate}
		%--------
		\end{itemize}

	\subsection{Group 3 Instructions}
		\begin{itemize}
		%--------
		\item Encoding:  \texttt{1010 aaaa bbbb oooo}
			\begin{itemize}
			%--------
			\item \texttt{a}:  general-purpose register \texttt{rA} or
				special-purpose register \texttt{sA}
			\item \texttt{b}:  general-purpose register \texttt{rB} or
				special-purpose register \texttt{sB}
			\item \texttt{o}:  opcode
			%--------
			\end{itemize}
		\item Instruction List:
			\begin{enumerate}
			%--------
			\item \texttt{ldb rA, [rB]}
			\item \texttt{ldsb rA, [rB]}
			\item \texttt{stb rA, [rB]}
			\item \texttt{ldh rA, [rB]}

			\item \texttt{ldsh rA, [rB]}
			\item \texttt{sth rA, [rB]}
			\item \textit{ldr sA, [rB, sp]}
			\item \texttt{str sA, [rB, sp]}

			\item \texttt{cpy rA, rB}
				\begin{itemize}
				%--------
				\item Copy \texttt{rB} to \texttt{rA}
				%--------
				\end{itemize}

			\item \texttt{cpy rA, sB}
				\begin{itemize}
				%--------
				\item Copy \texttt{sB} to \texttt{rA}
				%--------
				\end{itemize}
			\item \texttt{cpy sA, rB}
				\begin{itemize}
				%--------
				\item Copy \texttt{rB} to \texttt{sA}
				%--------
				\end{itemize}
			%\item \texttt{cpy sA, sB}
			%	\begin{itemize}
			%	%--------
			%	\item Copy \texttt{sB} to \texttt{sA}
			%	%--------
			%	\end{itemize}

			\item \texttt{reti}
				\begin{itemize}
				%--------
				\item Return from an interrupt, enabling interrupts
					(by setting the low bit of \texttt{ie} to \texttt{0b1})
					and jumping to the program counter value at
					\texttt{ira}.
				%--------
				\end{itemize}


			\item \textit{zeb rA, rB}
				\begin{itemize}
				%--------
				\item Zero extend the low 8 bits of \texttt{rB} and write
					the result into \texttt{rA}.
				%--------
				\end{itemize}
			\item \textit{zeh rA, rB}
				\begin{itemize}
				%--------
				\item Zero extend the low 16 bits of \texttt{rB} and write
					the result into \texttt{rA}.
				%--------
				\end{itemize}
			\item \textit{seb rA, rB}
				\begin{itemize}
				%--------
				\item Sign extend the low 8 bits of \texttt{rB} and write
					the result into \texttt{rA}.
				%--------
				\end{itemize}
			\item \textit{seh rA, rB}
				\begin{itemize}
				%--------
				\item Sign extend the low 16 bits of \texttt{rB} and write
					the result into \texttt{rA}.
				%--------
				\end{itemize}
			%--------
			\end{enumerate}
		%--------
		\end{itemize}

	\subsection{Group 4 Instructions}
		\begin{itemize}
		%--------
		\item Encoding:  \texttt{1011 iiii iiii iiii}
			\begin{itemize}
			%--------
			\item \texttt{i}:  12-bit signed immediate, \texttt{simm12}
			%--------
			\end{itemize}
		\item Instruction List:
			\begin{enumerate}
			%--------
			\item \texttt{bl simm12}
				\begin{itemize}
				%--------
				\item This instruction sets \texttt{lr} to \texttt{pc + 2}
					and jumps to the address
					\texttt{pc + (to\_s32(simm12) << 1)}.
				%--------
				\end{itemize}
			%--------
			\end{enumerate}
		%--------
		\end{itemize}

	\subsection{Group 5 Instructions}
		\begin{itemize}
		%--------
		\item Encoding:  \texttt{1100 aaaa bbbb cccc}
			\begin{itemize}
			%--------
			\item \texttt{a}:  general-purpose register \texttt{rA}
			\item \texttt{b}:  general-purpose register \texttt{rB}
			\item \texttt{c}:  general-purpose register \texttt{rC}
			%--------
			\end{itemize}
		\item Instruction List:
			\begin{enumerate}
			%--------
			\item \texttt{ldr rA, [rB, rC]}
			%--------
			\end{enumerate}
		%--------
		\end{itemize}

	\subsection{Group 6 Instructions}
		\begin{itemize}
		%--------
		\item Encoding:  \texttt{1101 aaaa bbbb cccc}
			\begin{itemize}
			%--------
			\item \texttt{a}:  general-purpose register \texttt{rA}
			\item \texttt{b}:  general-purpose register \texttt{rB}
			\item \texttt{c}:  general-purpose register \texttt{rC}
			%--------
			\end{itemize}
		\item Instruction List:
			\begin{enumerate}
			%--------
			\item \texttt{str rA, [rB, rC]}
			%--------
			\end{enumerate}
		%--------
		\end{itemize}

	\subsection{Group 7 Instructions}
		\begin{itemize}
		%--------
		\item Encoding:  \texttt{1110 aaaa bbbb cccc}
			\begin{itemize}
			%--------
			\item \texttt{a}:  general-purpose register \texttt{rA}
			\item \texttt{b}:  general-purpose register \texttt{rB}
			\item \texttt{c}:  general-purpose register \texttt{rC}
			%--------
			\end{itemize}

		\item Instruction List:
			\begin{enumerate}
			%--------
			\item \texttt{add rA, rB, rC}
			%--------
			\end{enumerate}
		%--------
		\end{itemize}

	%\printbibliography[heading=bibnumbered,title={Bibliography}]

%--------
\end{document}
