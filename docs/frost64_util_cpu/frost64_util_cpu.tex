\documentclass{article}

\usepackage{pbox}
\usepackage{graphicx}
\usepackage{float}
\usepackage{fancyvrb}
\usepackage[T1]{fontenc}
\usepackage{lmodern}
\usepackage{setspace}
\usepackage[nottoc]{tocbibind}
\usepackage[font=large]{caption}
\usepackage{framed}
\usepackage{tabularx}
\usepackage{amsmath}
\usepackage{hyperref}
\usepackage{fontspec}
\usepackage[backend=biber,sorting=none]{biblatex}
%%\usepackage[
%%	backend=biber,
%%	style=ieee,
%%	sorting=none
%%]{biblatex}
%\addbibresource{project_refs.bib}

%% Hide section, subsection, and subsubsection numbering
%\renewcommand{\thesection}{}
%\renewcommand{\thesubsection}{}
%\renewcommand{\thesubsubsection}{}

%% Alternative form of doing section stuff
%\renewcommand{\thesection}{}
%\renewcommand{\thesubsection}{\arabic{section}.\arabic{subsection}}
%\makeatletter
%\def\@seccntformat#1{\csname #1ignore\expandafter\endcsname\csname the#1\endcsname\quad}
%\let\sectionignore\@gobbletwo
%\let\latex@numberline\numberline
%\def\numberline#1{\if\relax#1\relax\else\latex@numberline{#1}\fi}
%\makeatother
%
%\makeatletter
%\renewcommand\tableofcontents{%
%    \@starttoc{toc}%
%}
%\makeatother

\newcommand{\respacing}{\doublespacing \singlespacing}
\newcommand{\code}[2][1]{\noindent \texttt{\tab[#1] #2} \\}
\newcommand{\skipline}[0]{\texttt{}\\}
\newcommand{\tnl}[0]{\tabularnewline}

\begin{document}
%--------
	\font\titlefont={Times New Roman} at 20pt
	\title{{\titlefont Example}}

	\font\bottomtextfont={Times New Roman} at 12pt
	\date{{\bottomtextfont} \today}
	\author{{\bottomtextfont Andrew Clark}}

	\setmainfont{Times New Roman}
	\setmonofont{Courier New}

	\maketitle
	\pagenumbering{gobble}

	\newpage
	\pagenumbering{arabic}
	%\tableofcontents
	%\newpage

	%\doublespacing

%\section{Abstract}
	%\setcounter{section}{-1}

\section{Table of Contents}
	\tableofcontents
	\newpage

\section{Registers, Main Widths, etc.}
	\begin{itemize}
	%--------
	\item The main width of the processor is 32-bit, contrary to what the
		name of the processor suggests.  Addresses are 32-bit. 
	\item General Purpose Registers
		\begin{itemize}
		%--------
		\item \texttt{u0}
		\item \texttt{u1)
		%--------
		\end{itemize}
	%--------
	\end{itemize}

	%\printbibliography[heading=bibnumbered,title={Bibliography}]

%--------
\end{document}
